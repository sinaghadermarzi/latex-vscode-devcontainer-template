\documentclass[12pt]{article}
\usepackage{simple_draft}

% Customize the running header here (appears on all pages after the first)


\title{Progressive Linkage}
\author[1,2]{First Author}
\author[2,3]{Second Author}
\author[2]{Third Author}
\affil[1]{First Department, First University, First Country}
\affil[2]{Institute Name, Second Country}
\affil[3]{Third Department, Third Affiliation, Third Country}
\date{January 2026}

\runninghead{Progressive Linkage, Jan 2026}

\begin{document}

\maketitle

\begin{abstract}
This template is required for scholarly articles and interviews, and it provides formatting guidance for artistic and hybrid submissions. We welcome creative formats and hybrid works that may not conform to these norms; however, we ask that all authors use this file as the basis for submission in order to support consistent technical processing and peer review. If you anticipate major deviations in format or media type, please contact the editorial team with clear suggestions in advance. \\
\textbf{Keywords:} keyword 1, keyword 2, keyword 3, keyword 4, keyword 5, keyword 6 
\end{abstract}

\section{Submission Guidelines}

If your proposal was accepted and you are submitting your final materials, please submit your work using the Open Journal System (OJS). Click \href{https://orcid.org/signin?client_id=APP-24WVJE2V6007Z3NV&response_type=code&scope=openid&redirect_uri=https:%2F%2Fjournals.library.columbia.edu%2Findex.php%2Fopenwork%2Fopenid%2FdoAuthentication%3Fprovider%3Dorcid}{here} to submit your materials. 

You will first be prompted to either register for an ORCID ID account (if you do not already have one) or authorize Columbia University Libraries Journals to access your existing ORCID record. For detailed instructions on how to register for an ORCID ID account and link your account to Columbia University Library, click \href{https://journals.library.columbia.edu/public/site/tutorials/OJS-orcid-tutorial.pdf}{here}. 

An abstract (no references) is required for all submissions. The abstract should serve as a brief description of the work and should include the artist's aims, details about the creative process, and other information relevant to the work and how it dialogues with the issue's themes. This should be no more than 250 words. Please also include a list of 5-6 keywords. All identifying information about the authors must be removed. 

Uploaded files (which includes article text, illustrations, and/or other media) cannot exceed 100MB. If upload limits are exceeded, a PDF may be uploaded that contains clearly labeled links to file hosting sites like Dropbox, Google Drive, etc. Make sure that all links are publicly accessible and will remain active during the peer review process. All identifying information about the authors must be removed from the titles and metadata of any uploaded files. In addition, files must be named descriptively (e.g. ``fig2\_SoundDiagram.png'' instead of ``image1.png'').

\subsection{Articles and Interviews}

\begin{itemize}[noitemsep]
\item Texts should be submitted as a Microsoft Word document with a .doc or .docx extension. 
\item Figures, illustrations, and media must be uploaded separately in addition to appearing within the text.
\end{itemize}

\subsection*{Artworks}

\begin{itemize}[noitemsep]
\item \uline{For Music:} The audio file for your submission in MP3 format.
\item \uline{For Poetry:} The text as a Microsoft Word document with a .doc or .docx extension.]
\item \uline{For Video:} The video file for your submission in MP4 format.
\item \uline{For Still Images:} The file for your submission in PNG, JPEG, SVG, or TIFF format, and of a suitable standard to reproduce, at least 300 dpi.
\end{itemize}

\subsection{Illustrations and Media}

For figures or illustrations, please upload them as PNG, JPEG, SVG, or TIFF files, and of a suitable standard to reproduce, at least 300 dpi. Full art preparation guidelines can be found \href{https://drive.google.com/file/d/1XBrYgYhOPmB3yAnfSL3PqGKXPNw0NAFE/view}{here}.

If the images are not the work of the author, please begin acquiring images and clearing rights as soon as your article or review has been accepted. Permission instructions can be found \href{https://drive.google.com/file/d/1TJF59MuoX3erJv5WXY6jg1Uh0hZzYU6N/view}{here}; a template for requesting permissions is available \href{https://drive.google.com/file/d/1MJ9yFqoR7GNSpBaNPJLexXKvkF4yifzR/view}{here}.

All images should be cleared for copyright purposes, and all costs and permissions to reproduce are the responsibility of the authors to provide correct credit lines for images and digital files. Online rights must be obtained from the rights-holder. openwork illustrates its digital publication in full color, although grayscale images may be accepted depending on the source material.

The Journal encourages multimedia additions to its articles. Please submit a short comment to the editors describing any plans for incorporating media if not already clear from submitted materials.

Copyright clearance and any associated costs for permissions for audio and video recordings, or other varieties of digital media, are the sole responsibility of the author.

\section*{Article-Specific Preparation Guides}

All texts should be original. Neither previously published work nor work under consideration for publication elsewhere will be accepted. Texts must be submitted in English.

\subsection{Style}

\subsubsection{House Style}
Please adhere to the following house style preferences.
\begin{itemize}[noitemsep]
    \item All text should be single-spaced.
    \item Paragraphs should be indicated by a line space, not indented or separated using paragraph spacing.
    \item Boldface should not be used except for headings, and underlining should not be used.
    \item In the main text, italicize titles of books and journals (please do not use underlining).
    \item Put titles of articles, doctoral theses, and exhibitions in single inverted commas.
    \item Figures and illustration references should be included in the text in parentheses (e.g., “(Fig. 2)”), and the captions should be numbered by order of appearance (not descriptively).
	\item Notes should be formatted as endnotes, with Arabic numerals.
    \item spelling should conform to American English standards (e.g., color, center, recognize, etc.)
    \item Quotations should be set within double inverted commas, and quotations within quotations in single inverted commas. Quotations more than six lines long should start on the next line and should be indented.
\end{itemize}

\subsubsection{Bibliography}
Articles must include citations to the appropriate peer-reviewed or preprint publications where applicable. References and in-text citations should follow the Chicago author-date format (cf. The Chicago Manual of Style, 17th edition). Examples include single-author citations of a book \parencite{tymoczko_geometry_2011} or a book chapter \parencite{spiegel_thoughts_2018}, multiple citations at a time \parencite{schoenberg_theory_2010, wyschnegradsky_manual_2017}, and a multiple-author citation of a journal article \parencite{michon_mobile_2017}. The complete bibliography should be included in a section entitled “References” at the end of the article. We strongly recommend using Zotero to manage your bibliography and automatically format the references and in-text citations to follow the Chicago author-date format.

\subsection{Formatting}

Please adhere to the following specifications for page layout and formatting by using the following template. Do not modify the template layout! Do not modify the line spacing!

\subsubsection{Layout}
\begin{itemize}[noitemsep]
    \item Paper size must be standard US Letter.
    \item Text must be written in one column. Multi-column layouts should not be used.
    \item Margins should be 1 inch on all four sides.
    \item Page headers must include a running head with the first author after the first page. 
    \item Page footers must include the page number right-aligned on all pages. 
\end{itemize}

\subsubsection{Section headings}
All headings must be left-aligned with boldface. Section headings should be 16 points and written in title case (i.e., every major word in the heading is capitalized). Sub-section headings should be 14 points and written in title case (i.e., every major word in the heading is capitalized). Sub-sub-section headings should be 12 points and written in sentence case (i.e., the first word is capitalized and the rest are in lowercase), with no extra line space separating the heading from the first paragraph. No more than three levels of headings should be used.

\subsubsection{Fonts}
The main text must be written using Times New Roman font. Standard body text should be 12 points, while the title should be 18 points. Other font types may be used if needed for special notations or equations.

\subsection{Figures}

Figures must be centered on the page. Figures that span the width of the page must be included on the top or bottom of the page. A minimum font size of 8 points is recommended for all text within figures. Please do not use stipple fill patterns as they do not render correctly in Adobe PDF.

\subsection{Tables}

Tables must be centered on the page. A minimum font size of 8 points is recommended for all text within tables. An example of a table is shown below.
\begin{table*}[hbt!]
    \centering
    \begin{tblr}{c|c}
        \toprule
        \textbf{Note Name} & \textbf{Frequency (Hz)} \\
        \midrule
        F4 & 349.23 \\
        \midrule
        G4 & 392 \\
        \midrule
        A4 & 440 \\
        \midrule
        B4 & 493.88 \\
        \midrule
        C5 & 523.25 \\
        \bottomrule
    \end{tblr}
    \caption{Note names and their corresponding frequencies.}
\end{table*}

\subsection{Captions}

Captions should follow each figure or table and conform to the guidelines below:

\begin{itemize}[noitemsep]
    \item Please number figures and tables in order and make sure that image files can be clearly identified from their filenames.
    \item Images not created by the author must have a catalogue-style caption supplied as a separate list. Captions should include:
    \begin{itemize}[noitemsep]
        \item [$\circ$] Maker
        \item [$\circ$] Title (in italics)
        \item [$\circ$] Date
        \item [$\circ$] Medium
        \item [$\circ$] Measurements
        \item [$\circ$] Location (Institution, Geographic Location)
        \item [$\circ$] Copyright statement
        \item [$\circ$] Courtesy line and/or photo credit
    \end{itemize}
\end{itemize}

An example of a caption is shown below: 

\textbf{Fig 1.} Yinka Shonibare, The Swing (after Fragonard), 2001, mannequin, cotton costume, 2 slippers, swing seat, 2 ropes, oak twig and artificial foliage, 3.3 x 3.5 x 2.2 m (Tate, London) © Yinka Shonibare; image courtesy Stephen Friedman Gallery, London.

\subsection{Equations}

Equations should be placed on separate lines and numbered. An example is shown below.

\begin{equation}
f(x) = a_0 + \sum_{n=1}^{\infty} \left( a_n \cos\frac{n\pi x}{L} + b_n \sin\frac{n\pi x}{L} \right)
\end{equation}

\section{Submission Checklist}

Prior to submitting your proposal to openwork, please verify that you have completed all of the items in the checklist below.

\begin{itemize}[noitemsep]
    \item The submission adheres to the Submission Guidelines and Article-Specific Preparation Guidelines (if applicable) outlined in the sections above.
    \item All identifying information (name, affiliation, etc.) is removed from texts, metadata, and associated media.
    \item Where available, URLs for the references have been provided.
    \item The author accepts responsibility for obtaining permission to publish any media submitted per the Submission Guidelines.
    \item The submission has not been previously published, nor has it been before another journal for consideration.
    \item The author grants the Journal the authority to publish the article as an Open Access article distributed under the terms of \href{https://creativecommons.org/licenses/by-nc-nd/4.0/}{Creative Commons Attribution, NonCommercial, NoDerivatives license (CC BY-NC-ND).} 
\end{itemize}

\printbibliography

\end{document}